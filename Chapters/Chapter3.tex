\chapter{Related Works}
\label{chap:related}

There has been several serious game for hemiplegic rehabilitation developed in the last few years. Similar to Hammer and Planks, these games also has some visualization feature which shows how the player performed so that the therapist is able to make the correct diagnosis. Thus, in this chapter, I first review some of these visualization. Then, since the nature of the input data is time series and movement data, I present some work in visualization which are related to this type of data.

\section{Visualization of Serious Game Result} 

Game result visualization is an integral part of a serious game used for rehabilitation since it's the feature which influence the accuracy of therapist analysis. Most serious game have an analytic feature, however the type of analysis presented depends on the nature of the game and the framework used in the rehabilitation. Therefore, for the purpose of this thesis, I only focus on reviewing serious game which are directed to hemiplegic patients rehabilitation.

In his paper, \cite{rahman} presents a rehabilitation framework for hemiplegic patients which combines the use of Kinect and LEAP\footnote{\url{https://www.leapmotion.com/product/desktop}} hand-tracking devices. These devices are attached to a 3D based game environment which was set to accommodate a set of primitive therapy motion such as forearm pronation/supination, shoulder and hip joint adduction/abduction, etc. Similar to Hammer and Planks, one of the game used in the framework requires user to navigate a plane by moving the hand to the right and left (hand-elbow flexion-extension). The recorded movement is then presented in line chart depicting the range of axis of elbow joint (180 degrees when fully extended and 20 degrees when fully flexed) over number of frames captured. Similarly, current visualization in Hammer and Planks also uses line chart to show average body movement over time. At first, line chart is used to represent Hammer and Planks gameplay, however in the end this approach is abandon since it's not intuitive enough. Details of this attempt can be found in chapter 4.

\begin{figure}
\centering
\includegraphics[width=110mm]{rahman_viz.png}
\caption{Visualization used in \cite{rahman} depicting the degree of forearm movement overtime}
\end{figure}

\begin{figure}
\centering
\includegraphics[width=150mm]{rahman2_elbowangle.png}
\caption{Visualization used in \cite{rahman2} depicting the speed of movement (m/s) of forearm overtime}
\end{figure}

In \cite{green}, a virtual reality rehabilitation system for children with hemiplegia was developed using TUI\footnote{\url{https://en.wikipedia.org/wiki/Tangible_user_interface}}. The game itself is displayed on LCD and the player interact with the game by placing the TUI on top of moving targets shown on the LCD. In this system, performances are measured by speed, accuracy and trajectory(mean movement efficiency). However, unlike \cite{rahman}, this system doesn't provide an interface in which therapist can analyse the gameplay.

Similar to Hammer and Planks, \cite{rahman2} introduced a framework which uses Kinect attached to Second Life\footnote{\url{http://secondlife.com/}} serious game environment. The mission of the game is to follow a set of movement that have been configured beforehand by the therapist. During the game, the movement of each body joint  is recorded and saved in Session Recorder. Afterwards, a Kinematic Analytic component will process this data and visualize the quality of improvement metrics of each body joint movement. Each metric is visualized with a dotted line chart over time as shown below. Even though it's possible to  see which line curve indicate an elbow flexion or extension, the therapist needs to count the number of the curve manually. This is not very efficient when the session is longer and there are more curve to count.

\section{Visualization of Time Series Data}

Since one of the requirement of the interface is to have the information of movement evolution over time, it is interesting to review how a time series data is usually visualize. \cite{aigner} discuss at length about the techniques of time series data visualization. This section, reviews some of these interesting techniques.

Considering that the recorded gameplay data contains spatial information (location of an event happened on the screen), some of the reviewed techniques are concerning visualizing spatio-temporal data. \textbf{Flow Map} depicts movements of object over time. Object movements are usually represented by directed trajectories over spatial space(i.e: map) with different color, width, angle of trajectories represent additional information. In order to overcome overlapping trajectories for huge amount of data, usually aggregation techniques (clustering, self organizing map, etc.) are introduced to group similar data point. Figure 3.3 shows an example of flow map depicting photographers movement between cities in Germany \cite{adrienko}. In this case, the aggregation considers three parameters: initial location, destination location, and time period in which the movement happened. Trajectories width indicate the number of photographers who move between the cities. Another visualization technique worth to mention is \textbf{Spatio-Temporal Event Visualization} which uses the space-time cube concept. In this concept, the x and y axis usually represent two spatial dimension while the third axis represent temporal dimension. The events are then represented as graphical objects which are mapped to the space-time cube location. Different events attribute can be represented in different size, colors, shape, or texture. Figure 3.4 shows space-time cube which depict convective clouds \cite{turdukulov}, human health \cite{tominski} and earthquake events \cite{gatalsky} from left to right. As we can see, the  spatial dimension of the left chart is area in pixel while the middle and right chart is a map. The events on the chart are represented with sphere objects with different color and different sizes. Even though space-time cube can portrays the spatio-temporal data, it has some downside. When there are too many events, occlusion is inevitable. It should be coupled with an appropriate interaction technique to allow users see the data from different perspective.

\begin{figure}
\centering
\includegraphics[width=90mm]{aigner_flowmap.jpg}
\caption{Flow Map}
\end{figure}

\begin{figure}
\centering
\includegraphics[width=150mm]{aigner_spatiotemporaleventviz.png}
\caption{Spatio-Temporal Event Visualization}
\end{figure}

One example of time-series visualization technique which doesn't concern spatial data is Theme River. First introduced in \cite{havre}, Theme River is used to visualize thematic changes over time of document collection. Each theme is represented as different colors which flows from left to right with different width over different time point. The width depicts theme strength over temporal axis. The purpose of this technique was to easily understand the evolution of theme strength over time. Figure 3.5 shows an example of Theme River representation of 1990 Associated Press newswire data. It can be seen on the chart that the theme baghdad, saddam, iraq, and kuwait are gaining strength  around the time Iraq invaded Kuwait on August 2, 1990. By following the flow of a certain color (theme) we can easily see the changes in theme strength and associate it with the events that affects the changes. Theme River should be supported with interaction techniques which allow user to rearrange river positioning over horizontal axis. 

Consequently, the Theme River technique is chosen due to its ability to show evolution of a certain data variable over time. Further details on the implementation can be found on Chapter 4.

\begin{figure}
\centering
\includegraphics[width=150mm]{havre_themeriver.png}
\caption{Theme River}
\end{figure}


\section{Visualization of Movement Data}

There has been numerous method and application developed to analyse movement data. In 

-- Andrienko's paper and book: related to movement data in general
-- MotionExplorer: related to body movement data



\section{Data Visualization Tool}
\subsection{D3.js}
general explanation of d3js and some example of how it is used to visualize time series and movement data.

\subsection{Three.js}
general explanation of three.js and some example.

