\chapter{Case Studies}
To evaluate the visualization functionality developed in this thesis, two case studies will be presented. There are two types of data collected: log files from healthy person who played the game and log files from patient. The log files from healthy person are gathered in duration of three weeks with each session played in different day. The patients' data are collected by NaturalPad\footnote{\url{http://www.naturalpad.fr/}}. Unfortunately, it is not possible to get information concerning patients' pathology due to confidentiality reason. For figures referred in this chapter, please see \ref{chap:appendix1}.

\section{Case Study 1: Normal Player}
The first case study is based on log files of game played by my colleague over the course of three weeks. In total, there are 20 sessions of BODYTILT game collected. Figure \ref{fig:app1_stacked} presents a comparison between the very first and last session played by the player. It shows that in the first session, there are a lot of events missed (yellow area) on the far right and far left of the screen which indicate that the player doesn't move her body to that extent. From the peak of area between 0-5 x axis, it can be concluded that the player moves more to the right. The green and red area which indicates positive and negative events only appears around -10 to 10 x axis which indicates that player only moves around the middle of the screen\ref{t11}\ref{t12}. It is understandable for first session because usually player needs time to get use to the game and get the feeling of how far he/she should move his/her boy to reach/avoid an object. On the other hand, in session 20, the peak of the area are more spread out and there are even green area at the far right\ref{t11}\ref{t12} which indicates that the player is able to move to that extent. It seems that the player already has the feeling of how to play the game.

At first glance on Figure \ref{fig:app1_heatmap}, it is noticeable that there are more spot on the top part of the right heatmap (positive events) and less spot on the bottom part of the right heatmap (negative events). It also can be concluded that the player is able to control the boat well on the right chart since there are less spot with high speed of the negative events\ref{t13}\ref{t14}. Which means that the player are getting better on playing the game. 

Focusing on events for Enemy as shown in figure \ref{fig:app1_stacked_enemy}, at a glance we can see that there are more negative events and very little positive events on the first session. While on session 20, there are less negative events and more positive events\ref{t15}. This supports the conclusion made previously that the player has gotten better performance over time. Similar notes are depicted by figure \ref{fig:app1_heatmap_enemy}. Here, on the right heatmap, there are more negative events which happened at faster screen speed\ref{t16}. This supports the fact that first time player are usually confuse on which direction to move their body/hand to make the boat go faster or slower.

The summary of all the session can be seen in figure \ref{fig:case1_type1}. As we can see, the number of positive events are steadily increasing though fluctuate \ref{t21}\ref{t22}. On session 19, there are very small number of negative events on all sections which indicate an improvement in the gameplay. In figure \ref{fig:case1_type1_select}, we can see the evolution of the positive events \ref{t23}. Overall, the number are steady except for a certain session. However, on the right most section there are some positive events can be found which indicates that overtime, the player move more to the right. On the clustered version of the chart (figure \ref{fig:case1_type1_select_c}), three sections in the middle are merged together since it shows similar evolution \ref{t27}. From figure \ref{fig:case1_type2} we can conclude that the movement are more concentrated in the middle area of the screen [\ref{t24}\ref{t25}\ref{t26}].

\section{Case Study 2: Patient}
The second case study is based on log files of game played by Patient 6. There are 15 sessions which are played over the course of three weeks. However, these sessions are of two game type HANDPOINT (6 sessions) and BODYTILT (9 session). For this case study, only the sessions of HANDPOINT exercise will be discussed.



