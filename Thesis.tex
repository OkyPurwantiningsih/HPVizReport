\documentclass[a4paper,11pt,twoside]{ThesisStyle}

\usepackage{lmodern}
\usepackage{subfig}
\usepackage{enumitem}
\usepackage{url}
\usepackage{algorithm}
\usepackage{array}
\usepackage[pdftex]{adjustbox}
\include{formatAndDefs}

\begin{document}
%----------------------------------------------------------------------------------------
%	TITLE PAGE
%----------------------------------------------------------------------------------------
\let\cleardoublepage\clearpage 
\begin{titlepage}
\begin{figure}
\vspace{-2cm}
   \begin{minipage}[r]{0.46\linewidth}
   \hspace*{-1cm}
      \includegraphics[scale=1.5,  width=.7\textwidth]{logocs.jpeg}
   \end{minipage} 
   \begin{minipage}[l]{0.46\linewidth}
   \hspace*{3.7cm}
      \includegraphics[scale=1, width=.5\textwidth]{it4bi_logo.jpg}
   \end{minipage}
\end{figure}
\vspace*{0.5cm}
\begin{center}
\noindent \Huge \textbf{Decision Support and Business Intelligence} \\
\vspace*{0.4cm}
\noindent \Large \textit{\textbf{ Information Technologies for Business Intelligence}}\\
\vspace*{0.4cm}
\noindent \Huge \textbf{Master Thesis} \\

\vspace*{0.8cm}
\noindent \LARGE Oky  \textsc{Purwantiningsih} \\
\vspace*{0.5cm}
\begin{tabular}{c}
\hline\\
\noindent {\Huge \textbf{Visual Analytics }} \\
\noindent {\Huge \textbf{on Human Body Movement Data}} \\
\noindent {\Huge \textbf{Applied on Healthcare}} \\
\\
\hline\\
\end{tabular}
\vspace*{0.2cm}
\\
\noindent {\Large prepared at Laboratoire d'Informatique, de Robotique et de Micro\'{e}lectronique de Montpellier and Universit\'{e} Paul Val\'{e}ry Montpellier} \\
\vspace*{0.2cm}
\noindent \large Defended on September ?, 2015 \\
\vspace*{1.5cm}
\end{center}
\begin{center}
\noindent \large 
\begin{tabular}{p{2.4cm}lcll}
      \textit{Advisor :}	& Arnaud \textsc{Sallaberry}		& - & LIRMM& arnaud.sallaberry@lirmm.fr\\
      & Jer\^{o}m\'{e} \textsc{Az\'{e}}		& - & LIRMM& jerome.aze@lirmm.fr\\
      \textit{Supervisor :}	& Nac\'{e}ra \textsc{Bennacer}		& - & Centrale Sup\'{e}lec& nacera.bennacer@supelec.fr\\
       \textit{In collaboration with :}	& S\'{e}bastien \textsc{Andary}		& - & NaturalPad & sebastien@naturalpad.fr\\
\end{tabular}
\end{center}
\end{titlepage}
\sloppy

\titlepage


%----------------------------------------------------------------------------------------
%	ABSTRACT
%----------------------------------------------------------------------------------------

\begin{vcenterpage}
\noindent\rule[2pt]{\textwidth}{0.5pt}

\noindent{\large\textbf{Abstract:}}
Serious game is used in healthcare to perform patients' rehabilitation and training. An example of such game is Hammer and Planks which is used to rehabilitate patient with balance disorders. Utilizing motion sensing input devices (Kinect, Wii Balance Board), the game is played by moving player's body. However, it is difficult to assess the rehabilitation progress from the game. In this Master thesis we design and develop visualization interface to help healthcare professional make correct diagnostic of patients' progress and therefore enable them to set the difficulty level for future rehabilitation session.
\\
To achieve this objective, we developed a visualization application which enables healthcare professional to analyse gameplay from two different views: (i) Session Visualization which allows analysis on one session. With this view, users are able to identify the frequency of movement related to objects on the game (ii) Summary Visualization which allows movement analysis over several sessions. This view enables users to navigate and explore the evolution of movement throughout all of the sessions. Here we propose a clustering method based on hierarchical clustering to group similar movement pattern over sections of game horizontal x-axis. This view also enables users to analyse in which area of x-axis the movement frequently happened. The proposed visualization is illustrated with two case studies which demonstrate the ability of the application to assess the rehabilitation progress.
\\
{\large\textbf{Keywords:}}
Information Visualization, Visual Analytic, Movement Analysis, Hierarchical Clustering
\\
\noindent\rule[2pt]{\textwidth}{0.5pt}
\end{vcenterpage}

\dominitoc

\pagenumbering{roman}
\cleardoublepage
%----------------------------------------------------------------------------------------
%	ACKNOWLEDGMENTS
%----------------------------------------------------------------------------------------

\section*{Acknowledgments}

for last

%----------------------------------------------------------------------------------------
%	TABLE OF CONTENTS, LIST OF FIGURES, LIST OF TABLES
%----------------------------------------------------------------------------------------

\tableofcontents
\newpage
\listoffigures
\newpage
\listoftables

\mainmatter
%----------------------------------------------------------------------------------------
%	CHAPTER
%----------------------------------------------------------------------------------------

\chapter{Introduction}
\label{chap:intro}

The advancement of gaming device technology such as Kinect, Wii Balance Board, Wii Remote, PlayStation Move, etc. has enabled players to control and interact with the game console through body movement. In healthcare, such technologies are used in serious game which can help users (doctors, patients, researchers, etc.)  perform health related activity such as patients' rehabilitation and training\cite{rahman,brezinka,green}. One example of such game is Hammer and Planks. Developed by NaturalPad\footnote{\url{www.naturalpad.fr}}, Hammer and Planks was designed to train the equilibrium of patient with balance disorders (specifically for hemiplegic people)\cite{diloreto}. A person with hemiplegic is paralized on one side of the body\footnote{\url{http://www.hemihelp.org.uk/hemiplegia/what_is_hemiplegia}}. Therefore, the gameplay is designed so that the player has to move their body to the right, left, forward and backward in order to train their affected side of the body. To support the purpose of rehabilitation, the healthcare professionals need to analyse the movement to  make a correct diagnostic of patients' progress and to adjust the difficulty level for the next rehabilitation session. In this thesis, we discuss the design of an interface to help healthcare professional understand the data generated from the game.

\section{Motivation}

Hammer and Planks is a vertical shooter game. The game world is in a 2D environment vertically scrolling from top to bottom in which a player navigate a ship from left to right through moving his body. It tells the story of a pirate named John K. One day a meteor fell down on John's ship and ruin it. There is a little left from his boat but it is still enough to build a new basic boat with what's left. While navigating his ship to collect driftwood/plank to upgrade it (hence the name Hammer and Planks), he also wants to find the ship which showered meteor and destroyed his ship. Therefore, in the game a player has to defeat all enemies which come on his way and he has to avoid being destroyed by bullets, reefs and other obstacles. Throughout the game the player has to collect bonuses (planks) to improve the ship. The game is usually played in short and intense phase and thus requires a lot of concentration\cite{diloreto}. Figure \ref{game_screenshot} shows the interface of the game. The boat in the middle represents the player, the reefs are obstacles which should be avoided, driftwood/planks shown in yellow circle are bonuses which should be collected, and the object emanating fires is an enemy which should be killed.

Currently, the game provides some charts which visualize player's body movement with respect to the horizontal axis and vertical axis (Figure \ref{current_viz}). However, the information that can be gathered from the visualization is not enough for the healthcare professional to be able to establish an informed diagnostic. It's hard to know how often the player move to the right or left. It's also not possible to know to which type of events (ie. avoiding an enemy, catching the bonuses) the movement is related to. Which is crucial since the therapist need to know if the player is able to develop strategy to play the game overtime. The existing visualization also provides chart to show the evolution of player's performance and total movement for all game sessions. However, the evolution of player's body movement is not depicted.

The purpose of this thesis is to address the problem mentioned by proposing a visualization interface to help healthcare professionals analyse gameplay data generated from the game so that they can determine the frequency and direction of player movement as well as movement pattern evolution over time. It is important for the interface to be simple and intuitive but at the same time still able to correctly shows the information needed by healthcare professionals. In this case, the visualization design process is difficult due to the heterogeneity and the size of the data. For instance, Hammer and Planks produces log file with tens of thousands records. Another challenge is how to visualize movement data that can be interpreted intuitively. Currently, there have been several approaches proposed on how to perform visual analytics on movement data. However, most of them deal with "geographical" movements \cite{adrienko_book} and only a few has been dealing with human body movement \cite{bernard2013}. It is also challenging to identify interesting movement pattern from the data as well as choosing the right approach to visualize changes of pattern.

\begin{figure}
\centering
\includegraphics[width=110mm]{hp_game.png}
\caption{Hammer and Planks Screenshot \label{game_screenshot}}
\end{figure}

\begin{figure}
\centering
\includegraphics[width=130mm]{current_viz.png}
\caption{Current visualization used in the game to show movements over horizontal and vertical axis \label{current_viz}}
\end{figure}


\section{Methodology}

To ensure that the visualization to be designed would satisfy the information needed by healthcare professionals, we followed the Nested Process Model proposed by Tamara Munzner \cite{Munzner:2009:NMV:1638611.1639181}. The model is divided into 4 levels: Domain Problem Characterization, Data/Operation Abstraction Design, Encoding/Interaction Technique Design, and Algorithm Design. These levels are nested; the output of a higher level will be the input for the lower level as shown in Figure \ref{munzner_model}.

\begin{figure}
\centering
\includegraphics[width=100mm]{munzner_model.png}
\caption{Munzner's Visualization Design Model with four nested layers \label{munzner_model}}
\end{figure}

In domain problem characterization level, the types of information needed by health professional from the visualization are defined. The output of this level would be a list of tasks that need to be solved by the visualization application. We then identify data structures which can support these tasks in the Data/Operation Abstraction Design level. In the third level, a good visualization and interaction techniques which can support the tasks are defined. At last in Algorithm Design level, an algorithm to support the visualization is proposed.

\section{Contribution}
In this thesis, we proposed a visualization interface for healthcare professionals to visualize Hammer and Planks gameplay data to help them analyse player movement during the game as part of rehabilitation process. The visualization provides two type of visualizations:
\begin{itemize}
\item A visualization which provides information on frequency and direction of body movement which is related to objects in the game for one game session.
\item An interactive visualization where healthcare professionals can analyse the evolution of player's body movement throughout all sessions. Here, we proposed an approach to easily navigate and highlight movement pattern. We also proposed a clustering algorithm based on hierarchical clustering to identify similar movement pattern. A distance function is defined to quantify movement pattern similarity which consider both movement evolution and proportion of movement frequency.
\end{itemize}

\section{Thesis Outline}

The remainder of this thesis is organized as follows. Chapter 2 discuss the domain problem characterization. Chapter 3 explores related work. The data abstraction is presented in Chapter 4. The Visual Mapping and Interactive Functionality of the proposed visualization are discussed in details in Chapter 5. Chapter 6 provides some case studies used to evaluate the approach and finally, chapter 7 concludes the thesis.

\chapter{Domain Problem Characterization}
\label{chap:problem}

In order to clearly understand the problems faced by the healthcare professional in interpreting the gameplay into a meaningful therapy routine, first it is important to have an understanding of how the game is played. Based on this understanding, then it will be possible to find out what kind of information needed by defining questions usually asked by the users. In the end, visualization requirement elicitation will be done by translating each question into list of tasks. This chapter discusses each one of these steps in details.

\section{Hammer and Planks Game Dynamics}

Played with Kinect, it is possible to play Hammer and Planks in three different ways:
\begin{enumerate}
  \item BodyTilt:
  Player puts both arm in his/her hips and move the the upper body (from the waist up) to the right, left, forward or backward to navigate the boat(figure \ref{fig:movement_type} (left)). 
  \item HandPoint:
  Player lifts one of his/her forearm in front of the body with the palm facing forward. Navigating the boat can be done by moving the forearm to the right, left, forward, or backward(figure \ref{fig:movement_type} (right)).
  \item ShoulderCGE:
  Player lifts one of his/her arm in front of the body and bend the elbow. Moving the elbow up and down will navigate the boat up and down the screen.
\end{enumerate}

For both the BodyTilt and HandPoint there are three direction available: (i) Horizontal: the screen scrolls from top to bottom and player navigates the ship from left to right (ii) Vertical: the screen scrolls from right to left and player navigates the ship from top to bottom of the screen(iii) Both: the screen scrolls from top to bottom and player navigates the ship from left to right. He/she can also move the boat faster or slower by bending the upper body (BodyTilt) or arm (HandPoint) forward or backward . For ShoulderCGE there is only vertical direction. In this thesis, I only interested in games played using BodyTilt and HandPoint movement for both direction since the information generated are richer, thus harder for the specialist to understand.

%\begin{figure}
%\centering
%\includegraphics[width=110mm]{mia_bodytilt.png}
%\caption{BodyTilt \label{overflow}}
%\end{figure}

\begin{figure}%
    \centering
    \subfloat[BodyTilt]{{\includegraphics[width=5.5cm]{cedric_bodytilt.jpg} }}%
    \qquad
    \subfloat[HandPoint]{{\includegraphics[width=5.5cm]{mia_handpoint.png} }}%
    \caption{Hammer and Planks movement type}%
    \label{fig:movement_type}%
\end{figure}


Before each session, the healthcare professionals will set the number of objects (enemies, bonuses, obstacles), activity duration and repetition, as well as area in which the objects can appear. Therefore he can adjust the difficulty of the game for different session.


\section{Target User Questions}

A traditional Hemiplegic therapy routine usually involved the therapist ordering a patient to perform several movement repetitively \cite{rahman}. By the end of the session, the therapist will analyse how the patient has performed based on the quality of movement as well as how the patient has progressed compared to the previous session. Based on this analysis, the therapist will then configure a new routine to further the patient's progress, if needed.

However, by using a game to facilitate the therapy, it is difficult to monitor how often the patient has moved his/her arm, to which direction and to which objects this movement is associated. Based on this reason, I identified 5 types of questions usually inquired:

\newcommand{\subscript}[2]{$#1 _ #2$}	
\begin{enumerate}[label=(\subscript{Q}{\arabic*})]
\item For a given session, to which direction (right/left) the player moved more? \label{q1}
\item For a given session, how does the player perform based on the number of objects collected, avoided, or killed with respect to the area of the movement?\label{q2}
\item For a given session, how does the player perform based on the number of objects collected, avoided, or killed with respect to the area of movement and the speed in which the game is played?\label{q3}
\item For a given patient, has he/she has improved in the game overtime?\label{q4}
\item For a given patient, has he/she has improved in a certain area overtime?\label{q5}
\end{enumerate}

\section{Visualization Requirements}

The gameplay of each game session is logged in a json file which contains information of the player, game setting, and every events (i.e. enemy killed, bonus collected, etc.) happened in the game. Based on these information and the question defined in the previous section, the tasks can be grouped into: task related to a session for a particular player (Task 1) and task related to the summary of a player which concerns all sessions (Task 2). The following are the tasks defined for each task group:
\newcommand{\task}[2]{$#1 #2$}
\begin{enumerate}[label=\textbf{(\task{T1.}{\arabic*})}]
\item \label{t11} visualize and compare the number of events of an event type at a given x area \ref{q1}\ref{q2}.
\item \label{t12} compare the number of events for different event type at a given x area  \ref{q1}\ref{q2}.
\item \label{t13} visualize and compare the number of events of an event type and its screen speed at a given x area \ref{q1}\ref{q2}\ref{q3}.
\item \label{t14} compare the number of events for different event type and its screen speed at a given x area \ref{q1}\ref{q2}\ref{q3}.
\item \label{t15} select and visualize the number of events for a certain object at a given x area \ref{q1}\ref{q2}.
\item \label{t16} select and visualize the number of events and its screen speed for a certain object at a given x area \ref{q1}\ref{q2}\ref{q3}.
\end{enumerate}

\newcommand{\test}[2]{$#1 #2$}
\begin{enumerate}[label=\textbf{(\test{T2.}{\arabic*})}]
\item \label{t21} visualize and compare the number of events of an event type among each session for one patient \ref{q4}\ref{q5}.
\item \label{t22} compare and navigate the number of events among different event type in a certain x area among each session for one patient \ref{q4}\ref{q5}.
\item \label{t23} select and visualize the number of events of a certain event type in a certain x area among each session for one patient \ref{q4}\ref{q5}.
\item \label{t24} visualize and compare the distribution of certain number of events of an event type over x area among each session for one patient \ref{q4}\ref{q5}.
\item \label{t25} compare and navigate the distribution of certain number of events among different event type over x area among each session for one patient \ref{q4}\ref{q5}.
\item \label{t26} select and visualize the distribution of certain number of events over x area for a certain event type among each session for one patient \ref{q4}\ref{q5}.
\item \label{t27} extract and visualize similar pattern of number of events over a certain x area and sessions for one patient \ref{q4}\ref{q5}.
\end{enumerate}

\chapter{Related Works}
\section{Serious Game in Healthcare}
explain how serious game is used in healthcare. discuss some example.
\section{Visualization of serious game result} 
discuss how the result of serious game are usually presented (couldn't find any specific paper discussing about this, but there are some paper about serious game which has some visualization to analyze the result of the game)
Discuss about state of the art game visualization

\section{Visualization of Time Series Data}
discuss visualization paradigm usually use to visualize time series data
\section{Visualization of Movement Data}
discuss paper about movement data visualization, ex: MotionExplorer, Andrienko's paper and book
\section{Stream Graph}
discuss examples of stream graph implementation, how it is used and for which kind of data

\section{Data Visualization Tool}
\subsection{D3.js}
general explanation of d3js and some example of how it is used to visualize time series and movement data.

\subsection{Three.js}
general explanation of three.js and some example.


\chapter{Data Abstraction}

In this chapter I discuss the design of data structure and clustering technique used to support the visual requirement. First, an overview of the input data generated from the game will be explained. Then a description on how this data is extracted to be the input of the visualization interface will be given. Finally, a clustering algorithm selected to be used in the visualization will be discussed.

\section{Game Events Structure}
\subsection{Event Category}
The goal of Hammer and Planks game is to kill all of the enemies while avoiding any attack from the enemies and obstacles\cite{diloreto}. Along the way, player can also catch bonuses to increase their score. Based on these, I identify three different objects within the game: Enemy, Bonus, and Obstacle. For each of these object, there are certain events associated. Each event which happened during the gameplay is recorded in the log file with the following information: event type, timestamp, object id, and location. In total, there are 8 event types:
\newcommand{\events}[2]{$#1 _ #2$}	
\begin{enumerate}[label=({\arabic*})]
\item Catch: when a bonus is catched
\item Miss: when a bonus is missed or player's attack on enemy is missed
\item Dodge: when an obstacle is avoided
\item Collision: when the player's boat collide with an enemy or obstacle
\item Kill: when an enemy is destroyed by player's boat
\item Hit: when the player's attack hit an enemy
\item Hurt: when the enemy's attack hit player's boat
\item Miss: when the enemy's attack missed player's boat
\end{enumerate}

Based on the level of impact of each event to the user's boat, we characterize the event by assign it with Positive, Neutral, or Negative as shown in the following table:

\begin{center}
    \begin{tabular}{| l | l | l | l |}
    \hline
    Events & Bonus & Obstacle & Enemy \\ \hline
    Positive & catch & - & kill,hit\\ \hline
    Neutral & miss & miss & dodge\\ \hline
    Negative & - & collision & hurt, collision\\
    \hline
    \end{tabular}
\end{center}

\subsection{Game World Coordinates}
Each object and event in the game are assigned with 3D location coordinates. An \textit{x} axis of this coordinate indicate horizontal axis of the screen. However, \textit{y} axis indicate vertical axis in the game world which means -\textit{y} is a location under the sea and +\textit{y} is above the sea. \textit{z} axis indicate vertical axis of the screen. The visualization uses the x axis to represent body movement over horizontal axis and z axis to calculate screen speed as explain in the following sub section.

\subsection{Screen Speed}
In the game, a big number of positive events indicate a good player's performance. However, it is important to consider whether the events happened when the player's boat move fast or slowly \ref{t14}\ref{t16}. Getting all the bonuses while moving fast requires precise hand/body movement which indicates improvement in rehabilitation process. Boat speed while navigating the sea is basically the speed in which the screen scroll($\upsilon_{scr}$). This is calculated by identifying the location(apparition z coordinate $\theta_{apr}$) and time (apparition time $\textit{t}_{apr}$) of an object when it first appear on the screen, and location(event z coordinate $\theta_{evt}$) and time(event time $\textit{t}_{evt}$) when an event happened on that object.

$$ \upsilon_{scr} = \frac{\theta_{evt}-\theta_{apr}}{\textit{t}_{evt}-\textit{t}_{apr}} $$

\section{Clustering Algorithm}

clustering model is also in this part to see what common evolution of section of the game
\chapter{Visual Mappings And Interactive Functionalities}
This chapter describes different visualizations and interaction methods developed based on the visualization requirements and data abstraction discussed in Chapter 2 and 4. In general, the visualization is divided into two parts: (i) Session Visualization which visualize movement in one particular session (for \textit{T}1.x) and (ii) Summary Visualization which visualize movement over different sessions (for \textit{T}2.x). Both visualization is organized in an application where user can select player and sessions he has played. At first, the earlier version of visualization will be explained. This earlier version is not used in the final version since it's difficult to get any information intuitively. Then, each type of visualization and its interaction used in the final version will be discuss. In the end, the application which encapsulate both visualization will be presented.

\section{Early version of Session Visualization}
The first visualization method chosen to represent \ref{t11} is line chart. In this approach, the x area is shown as a horizontal axis and the number of events shown in vertical axis with the line signify the changes of number of events for different x area unit. In the log file, each event is recorded with distinct 3D coordinate location. The x value from this coordinate is a decimal, therefore visualizing each one of this x value will require a lot of space. To solve this, the events are then grouped by the rounded x value. In Figure \ref{first_linechart} below, Negative events are shown in red line and Positive events are shown in blue line. As we can see, it's possible to know which event type happened more in a certain x unit, however it is difficult to see how big a percentage is it compare to total number of events happened in the same x unit.

\begin{figure}
\centering
\includegraphics[width=100mm]{linechart.png}
\caption{First visualization version for \ref{t11} using Line Chart}
\label{first_linechart}
\end{figure}

\begin{figure}
\centering
\includegraphics[width=100mm]{scatterplot.png}
\caption{First visualization version for \ref{t12} using Scatter Plot}
\label{first_scatterplot}
\end{figure}

Visualization method chosen to represent \ref{t12} is scatter plot. At first, each event type is presented in three different chart area: top are for Positive events, middle are for Neutral events, and bottom area for Negative events. Similar to the line chart, the x value from the 3D coordinate location are represented in horizontal x axis. However, the vertical axis here represents screen speed. For the scatter plot, each event is shown as a plot in the chart area according to it's x value and screen speed as shown in Figure \ref{first_scatterplot}. Even though similar pattern can be seen on the scatter plot, occlusion problem prevents users to know how many event are actually happened in a certain x area.

\section{Session Visualization}
The Session Visualization shows events within a game session. The requirement can be split into two: (i) knowing the distribution of events and movements (ii) knowing the distribution of events, movements and screen speed. Basically, (ii) is a detailed view of (i). Therefore, there are two chart developed to meet these requirements: stacked area graph for (i) and heatmap for (ii), each of which will be explain in details in this section.

\subsection{Stacked Graph}
Build on layered area graph, Stacked Graph is widely used to visualize evolution of variable over times such as document theme \cite{havre}, box office movie revenue\footnote{\url{http://www.nytimes.com/interactive/2008/02/23/movies/20080223_REVENUE_GRAPHIC.html?_r=0}}, listening history in Last.fm \cite{byron},etc. Stacked Graph is chosen because its ability to show individual value of a variable, the difference between values of different variables as well as the total of overall value. In our approach, instead of using this metaphor to show evolution over time, it is used to show distribution of events over spatial coordinate \ref{t11} as shown in Figure \ref{fig:linear}. Here, the horizontal axis represents x coordinate and vertical axis represents number of events. Each event type is represented as an area with different color: Red (Negative), Yellow (Neutral), and Green (Positive). However, with this approach it is difficult to see the trend for each individual events which is not on the base of the chart. \cite{alan} propose an interactive solution to this problem by sinking the selected category to the horizontal axis. Thus, in our visualization, different layout of stacked graph is provided: 
\begin{enumerate}
  \item Linear (Figure \ref{fig:linear}): zero y axis is used as the baseline, with the stack ordered from the bottom as Negative, Neutral, Positive.
  \item Silhouette (Figure \ref{fig:silhouette}): the graph is centered as in streamgraphs.
  \item Positive (Figure \ref{fig:positive}): zero y axis is located at the top of the chart and is used as the baseline with the stack ordered from the top as Positive, Neutral, Negative.
  \item Neutral-Negative (Figure \ref{fig:neutral-negative}): zero y axis is located in the middle of the chart. Neutral and Positive events are shown on the positive area of y axis and Negative events are shown on the negative area of y axis.
  \item Positive-Neutral (Figure \ref{fig:positive-neutral}): zero y axis is located in the middle of the chart. Positive events are shown on the positive area of y axis, while Neutral and Negative events are shown on the negative area of y axis.
\end{enumerate}

\begin{figure}
\centering
\includegraphics[width=100mm]{stackedgraph_linear.png}
\caption{Stacked Graph with Linear Layout}
\label{fig:linear}
\end{figure}

\begin{figure}[htp] % not h only
\centering
\subfloat[Silhouette]{%
\includegraphics[width=0.4\textwidth]{stackedgraph_silhouette.png}%
\label{fig:silhouette}%
}\hfil
\subfloat[Positive]{%
\includegraphics[width=0.4\textwidth]{stackedgraph_positive.png}%
\label{fig:positive}%
}

\subfloat[Neutral-Negative]{%
\includegraphics[width=0.4\textwidth]{stackedgraph_neutral_neg.png}%
\label{fig:neutral-negative}%
}\hfil
\subfloat[Positive-Neutral]{%
\includegraphics[width=0.4\textwidth]{stackedgraph_positive_neutral.png}%
\label{fig:positive-neutral}%
}

\caption{Different Layout of the Stacked Graph representing number of events over x-axis}
\end{figure}

For each stacked graph layout, user can choose which object type to show on the graph \ref{t13}. Options are available as radio button on top of the chart. Therefore, choosing Bonus will show only Positive and Neutral events, choosing Obstacle will show only Neutral and Negative events, and choosing Enemy will show all event type.

\subsection{Heat Map}
\begin{figure}
\centering
\includegraphics[width=130mm]{heatmap3.png}
\caption{Heatmap showing movement distribution over x-axis and screen speed}
\label{heatmap}
\end{figure}
Heat Map is a quite popular visualization method nowadays due to its ability which allows user to see variable with the highest value at one glance. Most of the time, heatmap is implemented on geographical map to represent variable value over certain area on map, i.e: Natural Disaster Risk by Location\footnote{\url{http://www.rms.com/}}, population density\footnote{\url{https://en.wikipedia.org/wiki/Population_density}}, Number of picture taken in an area\footnote{\url{http://sightsmap.com/}}, etc. Heat map is also used to track eye movement or mouse click on a website, and representing DNA microarray data in the form of cluster heat map\cite{friendly}. Heat map uses color gradation to represent the hotness level of a variable. Usually, red color is used to represent the high value (hot) and blue is used to represent the low value (cold). However, other color combination can also be used. To represent distribution of events and screen speed over x axis \ref{t12}, the events are first grouped based on their x values and normalized screen speed. The number of events is then represented as heat map on the graph with highest number of events in red color and the lowest number of events in blue color. Normalized screen speed is represented as vertical axis and x value is represented as horizontal axis. Using the same approach used in scatter plot chart, each event type is presented in different area: top for Positive, middle for Neutral, and bottom for Negative (Figure \ref{heatmap}). For Neutral events, the screen speed is not calculated since it basically mean an object has been avoided or missed. For Negative events, the screen speed is represented in negative to show that it's an uncalculated movement. For the heatmap, user can also choose to show a specific object \ref{t14} by clicking the radio button associated with the desired object.

\section{Summary Visualization}
The Summary Visualization fulfills the requirements concerning movement evolution over time (\textit{T}2.x). In this case, a single session is considered as a single time point. Since user are interested in evolution over a certain x-area, the visualization are divided into sections of x-area. There are three ways of division: by the range of x-area, by the number of events within an area, and by clustering. The following explains the three approaches and its interaction technique in details.

\subsection{Visualization by range of x-area}
To fulfill \ref{t21}, a streamgraph metaphor is chosen to show evolution of movement (represented by events) over time. 
Time is often displayed on x-axis, but here, x-axis represent the x-axis of the screen, so time is displayed on the y-axis. Some visualizations are also based on this approach, like Visual Sedimentation \cite{huron}. Here, sessions represent time with the earliest one shown at the bottom and the latest one shown on the top. The x axis is then divided into sections of the same range based on user input. For each section, events are then filtered to the one which happened within the section x boundary. The filtered events are grouped based on session number and event type. Number of events within this group is then presented in vertical streamgraph layout with event type represented using the same color used in the Session visualization (Figure  \ref{fig:type1_not_clustered}). 

Shown in Figure \ref{fig:type1_section} a section in the chart. Here, the lower x boundary is -20 and the upper x boundary is -7.33. Within this area, the evolution of events throughout all session can be seen\ref{t21}. It is also possible to see which session has the most or least number of events by comparing the total length of all event type in one session. 

\begin{figure}
\centering
\includegraphics[width=130mm]{summary_not_clustered.png}
\caption{Summary Visualization divided by range of x-area}
\label{fig:type1_not_clustered}
\end{figure}

\begin{figure}
\centering
\includegraphics[width=80mm]{section.png}
\caption{A section in Summary Visualization}
\label{fig:type1_section}
\end{figure}

\begin{figure}
\centering
\includegraphics[width=130mm]{summary_type2.png}
\caption{Summary Visualization divided by number of events}
\label{fig:type2}
\end{figure}

\subsection{Visualization by number of events}
This second type of Summary Visualization uses the same approach explained previously. However, a section is calculated based on the total number of positive and negative events instead of the range of x-axis \ref{t23}. Therefore, based on the distribution of positive and negative events, one section in the chart may have bigger x-range than the other section. Only by comparing the size of sections, it's possible to know in which area most of the events are concentrated. Figure \ref{fig:type2} shows that the events are more concentrated in the middle area of the screen. Here, it can be concluded that on the far right and far left of the screen, there are more neutral events compare to the middle area.

\subsection{Visualization by clustering}
This third type of visualization fulfills \ref{t25}. As explained in section 4.2, initially the chart is divided into sections with range equals to an x-axis unit. Depending on the threshold value inputted by user, consecutive sections with distance below the threshold will be merged. This process is repeated until there is no sections with distance below the threshold. The input threshold ranges from 0 to 1. Thus, when user inputted threshold = 0, none of the section will be merged. On the other hand, when user inputted threshold = 1, all of the sections will be merged. Figure \ref{fig:type3_not_clustered} shows when user input threshold = 0, while Figure \ref{fig:type3_clustered} shows threshold = 0.24. As we can see, there are some sections which are merged, indicated with bigger section size. 4 sections are merged together forming one section with x range between 16-20 and 12-16. On the left side of the chart, 7 sections are merged together forming one big section ranging from -13 to -20.

\begin{figure}
\centering
\includegraphics[width=150mm]{type3_not_clustered.png}
\caption{Summary Visualization divided by movement evolution similarity, clustered with threshold = 0}
\label{fig:type3_not_clustered}
\end{figure}

\begin{figure}
\centering
\includegraphics[width=150mm]{summary_clustered.png}
\caption{Summary Visualization divided by movement evolution similarity, clustered with threshold = 0.24}
\label{fig:type3_clustered}
\end{figure}

\subsection{Interaction Technique}
On top of the chart, an interaction bar is provided where user can interact with and change some variable in the chart. In first panel of the interaction bar (Figure \ref{fig:interaction_bar}), three sliders are provided: (i) slider to input the number of slices so that each section will have the same x-range (ii) slider to input the number of slices so that each section will have the same total number of positive and negative events (iii) slider to input threshold so that each section will have a cluster of similar movement evolution. When user uses slider (i) and (ii), the value on the slider defines the number of sections on the chart. For (i), the more number of sections, the smaller the x-range. While for (ii), the value selected on the slider is a denominator. The input is calculated by dividing total number of all positive and negative events by the value selected on the slider. Thus, the smaller the number chosen on the slider, the bigger the number of events. In the second panel, user has the options to choose which event type to show on the chart. This fulfills requirement \ref{t22} and \ref{t24}. By default all event type will be shown. 

Once the chart is generated, user has the ability to slide/drag the line between each session or the small triangle at the bottom of the line to the right or left(see Figure \ref{fig:type1_section}) to change the range of it's neighbouring sections\ref{t21}\ref{t23}. While dragging, the text on top of the line changes based on the current x value of the dragged line. When a line is dragged over another line, the two sections will be merged creating a new section with different x range. Therefore, user may be able to gain the information in which particular area a certain type of events starting to happened. Figure \ref{fig:line_dragging} shows changes on affected sections when line 11.63 is dragged to the left to position 10.7. We can see that Negative events on the circled area started to appears from x = 10.7. It is also possible to divide a section into two sections by clicking the top area of the chart in between the lower and upper section boundary text. This allows user to know the distribution of event type within a section.

\begin{figure}
\centering
\includegraphics[width=130mm]{interaction_bar.png}
\caption{Interaction Bar for Summary Visualization}
\label{fig:interaction_bar}
\end{figure}
\begin{figure}
\centering
\includegraphics[width=110mm]{line_dragging.png}
\caption{Navigating section line to highlight pattern}
\label{fig:line_dragging}
\end{figure}

\section{General Interface}
\begin{figure}[H]
\centering
\includegraphics[width=150mm]{interface_app_compare.png}
\caption{Application Interface}
\label{fig:app_interface}
\end{figure}
Both the Session and Summary visualization are attached into an application which allows user to navigate different player and session. Unlike the visualization interface, the general interface is built using extjs library which allows the development of desktop-like web application. The interface of this application is divided into two areas: Navigation area and Visualization area (Figure \ref{fig:app_interface}). In Navigation area, user can choose patient and the sessions they have played. On clicking a session, a Session visualization of this session will be shown on the visualization area. It is also possible to open more than one Session visualization and rearrange the visualization window to compare gameplay between sessions (similar to navigating multiple windows in desktop). On clicking patient's name, Summary visualization for the chosen session will be shown.

\chapter{Case Studies}
write a kind of stories. Looking at this visualization, I see this and that. This correspond to this task and this task.
find at least one example for all the task we've defined before. We can say: here, there is a difference between people with pathology and without pathology.
for ex: the movement are only in the middle for the people with pathology, while for normal people, there are movement on the side as well.
for ex: there are a lot of green in this part. there are less event at the beginning, when there are more event, then there are more red area.
\section{Normal Player}
\section{Patient}
\chapter{Conclusion}

In this thesis we presented ...
what has been discussed
  problem and user requirement
  
proposed approach and what it can do

future improvement


%----------------------------------------------------------------------------------------
%	APPENDICES
%----------------------------------------------------------------------------------------

%\appendix%

%\chapter{Appendix Example}
\label{chap:appendix1}

\section{Appendix Example section}

And I cite myself to show by bibtex style file (two authors).

This for other bibtex stye file : only one author \cite{diloreto} and many authors \cite{diloreto}.%

%----------------------------------------------------------------------------------------
%	BIBLIOGRAPHY
%----------------------------------------------------------------------------------------

\bibliography{Thesis}
\bibliographystyle{plain}


%\printnomenclature

%\cleardoublepage


\end{document}
